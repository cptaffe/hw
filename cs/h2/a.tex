\documentclass{article}

\usepackage{xltxtra}
\setmainfont{GFS Didot}

\title{$h_{2}$}
\author{Connor Taffe}

\begin{document}
  \maketitle

  \section{C-2.6}
    Three keys are not dusty, the passcode is four keys long.
    \begin{equation}
      3^4
    \end{equation}

  \section{C-2.11}
    Using an assymetric key is the most safe system. That way there is no danger in decrypting the code by reading a large number of known account numbers. Hashing the account number is very poor because the attacker can probably guess the hashing scheme.

  \section{R-3.3}
    \begin{equation}
      1+2+4=7
    \end{equation}
  \section{R-3.12}
    \begin{equation}
      200,000*2^{24}
    \end{equation}
  \section{R-3.13}
    \begin{equation}
      100*500,000
    \end{equation}
  \section{R-3.14}
    Yes, because his user has read and write permissions. Another member of group {\tt hippos} does not have read access though.
  \section{C-3.1}
    Yes it does, as the search space is now 1 instead of $2^l$ where l is the bit length of a random salt value.
  \section{C-3.2}
    $2^20=1048576$ vs $36^8=2821109907456$ for an 8 character password using alphanumeric characters. It is about $10^6$ times weaker than a subpar password.
  \section{C-3.3}
    Kind of, it is still $2^20$ since there are twenty pairs of which one is correct for a single login. But, since they are now in random order, the password cannot be stored as a single sequence of answers, so the computing power to check each pair is higher and requires a copy of each of the 20 favorite pictures to compare. So in practicality it is more difficult.


\end{document}
