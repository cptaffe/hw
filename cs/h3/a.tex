\documentclass{amsart}

\title{$h_{3}$}
\author{Connor Taffe}

\begin{document}
  \maketitle

  \section{r-4.2}
  The existance of user {\tt 12345} is a backdoor. The user was used by the logic bomb to login and gain permissions to run {\tt fix.exe} and {\tt purge} which deleted the files on Omega's manufacturing server.

  \section{r-4.13}
  Eve uses a trojan horse attack, disguising her spywaere as a normal USB driver. She uses a ``social engineering'' vector, placing a usb with a logo supposed to fool the finder into opening the usb on, she hopes, a company laptop.

  \section{c-4.4}
  The probability a computer is correctly diagnosed is ninety five percent. If it has been diagnosed as infected, the probability is as follows:
   \begin{equation}
     \frac{0.01*0.95}{(0.5*0.99)+(0.01*0.95)} = 0.01883052527254708
   \end{equation}

  \section{r-5.4}
  The sequence number for the {\tt ACK} is $156955004$, and the acknowledge number is $883790340$.

  \section{r-5.5}
  No, the address indicates the location of a resource. If two interfaces have the same address, routing will fail to function correctly and each will fial to function properly.

  \section{c-5.4}
  Encrypt a message with the shared key $k$ and then with the server's known public key $k_{p}$. If it is the real server, and we are actually sharing $k$ with it, it should be able to send us back the unencrypted message encrypted with it's private key $k_{s}$ to further ensure identity. We unencrypt the message with $k_{p}$ and verify it is the same as what we sent. If so, we have verified that this is both the server we want and that it knows about $k$.

  \section{c-5.7}
  The attacker can guess the next random number the client will produce and thusly intercept the TCP handshake.

  \section{c-5.15}
  The latency from Chicago to Copenhagen is much longer than 10ms.

  \section{r-6.2}
  $0.99 * 65536 = 64880.64$ requests per second.

  \section{c-6.12}
  $2^{16}$ tcp connections for sequential vs $2^{16}*ln(2^{16})$ connections for a random scan because it may choose a port it has already hit each random attempt. The real question is: why would you ever do a random port scan that didn't remove ports you'd already hit?

\end{document}
