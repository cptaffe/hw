\documentclass[11pt]{article}
\usepackage{amsmath,amssymb,amsthm}
\usepackage{fancyhdr}

% margins
\usepackage[vmargin=1in,hmargin=1.5in]{geometry}

% Config
%%%%%%%%%%%%%%%%%%%%%%%%%%%%%%%%%%%
\newcommand{\ass}{0}
\newcommand{\name}{Connor Taffe}
\newcommand{\tno}{3742} % last 4 digits of T number.
%%%%%%%%%%%%%%%%%%%%%%%%%%%%%%%%%%%

\title{
	$A_{\ass}$ \\
	{\large Assignment \ass\\
	CS 4383; Professor Bush}
}
\author{
	\name. T no. \tno
}

\pagestyle{fancy}
\rhead{Assignment \ass}
\lhead{{\name}. T no. \tno}

\begin{document}
\maketitle

The paper ``Computing Machinery and Intelligence'' was published in {\it Mind} in issue 49 in the year 1950 and describes what Turing calls the ``Imitation Game.'' Turing proposes that this game is a new form of the problem ``Can machines think?'' Instead of defining what it is to think, he decides to replace the question with a problem. It seems to be aimed to be an agreeable example of what we intuitively define thinking as. An acceptable synonym of the undefined ``strong AI.'' I propose that the Turing Test is a test to check the aptitude of a computer at the Imitation Game, and little more.

The Imitation Game proposes a game consisting of two people, a man and a woman, known to us as A and B respectively. A third participant, C, is the interrogator. Unbeknownst to participant C, an ordering of labels of X and Y are given to A and B. The labels are the only distinguishment C can make. Participant B is to assist C in labeling X and Y as A and B and thusly determining the sex of either. A is to deter C's progress. Since C initially does not know who is helping or hindering his progress, the game seems of some difficulty. Turing also notes that A and B communicate to C only though some teleprinter as not to give anything away by voice, looks, handwriting, etc. Thusly an attribute must be found out by communication only. He then alters the game and replaces A with a machine, later restricted to a digital computer.

The Imitation Game, and thusly the Turing Test, test only this single scenario of interaction. What generalizability it has is yet to be seen and even somewhat impossible to quantify. ``Computing Machinery and Intelligence'' later states the idea of a ``Human Computer,'' or a human and his ability to solve problems of the kind computers solve. Turing uses a human in the place of a computer system. His memory and paper records served as his memory; his instructions were english versions of CPU instructions similar to those like, for example, {\tt jmp} on Intel x86 platform CPUs. In this way, Turing suggests that a subset of human actions can be equivalent to computer actions.

I propose that all human actions are computable actions, as they were computed with a human mind. At some low level, there is a unit which has only a finite number of states, and which can be simply simulated to imitate en masse well enough the whole of the human consciousness. This of course may require a near perfect knowledge of the state of a human mind, but theoretically with this knowledge it could be achieved with some computer with finite resources (however large). It is not that hard to imagine that a less specific model of the human mind could achieve near enough similarity to the human mind. Therefore, a computer, eventually, will be able to complete the Imitation Game, as it is a subset of a humans abilities. Turing's paper notes that the idea is not that if any computer that now exists can complete the Imitation Game, but whether any computer ever will be able to, and I believe it can be well established that a computer eventaully can based on this previous argument.

Based on this, it can be concluded that the Imitation game tests one interaction that serves as what some percieve to be a significant subset of human interaction. And that since it is a susbset of computer interaction, it can be computed by a computer, because a theoretical computer is able to imitate all of a human's conciousness. This introduces a new use for the Imitation Game, not that of showing if it is possible at all, but if it is possible in some way, in some meaningful amount of time. As it is obvious that human actions can be completed with finite resources in finite time, because they are already completed in finite time and finite space by a human, the question now moves to will a computer ever be feasable that can compute enough information to compete in the Imitation Game in a human-esque amount of time. A caveat that is exploited regularly is that a computer need not be imitating a human's every mechanism to compete, but only some focused portions of a human's abilities.

The Turing Test is thusly a mechanism for showing if it is {\it practically} possible to imitate a human's involvment as participant A in the Immitation Game. This means that the computer must imitate human language. Human language seems complex to us, as it is not something that can be written simply in mathematics and seems to obey no simple finite number of rules. We have also studied the human brain and learned that humans devote a large portion of thier resources to communication as it is so evolutionarily advantageous. Thusly, we see it as difficult to simulate, and it has proven itself thusly as the field of Artificial Intelligence advances. It also tests how well the computer can imitate other parts of being human. It must understand many experiences it has never experienced, and be ready to respond to puns and references which it has not had the experience to understand. Thusly it also tests what body of social conciousness can be made avaliable for a computer. As this is most difficult, this point seems to draw away from the idea of whether a machine can think, and rather to can a machine have this infomormation avaliable.

The Turing Test also ignores the other aspects of thinking. Rather, he focuses on human communication. I believe that animals can ``think,'' and Turing seems to agree with this in his refutation of the counterclaim involving souls. We can easily see that the Imitation Game would go poorly for a fox or an elephant. Although they cannot write or spout human experiences, they are no less intelligent and can ``think.''

It is difficult to conclude that the Turing Test is any more useful than to test an algorithm's adeptness at the Imitation Game itself. We lack any grading to show beyond intuitive thinking that this test is applicable to anything outside of itself and its immediately needed aptitudes. Turing's original assertion that the Turing Test tests ``thinking'' is almost absurd in this light. I believe a test that would truly imitate ``thinking'' would have to create solutions to problem spaces as they arrive in the way that humans do, and I have not seen that the Imitation Game would make this necessary.


\end{document}
