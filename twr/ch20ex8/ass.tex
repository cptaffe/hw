\documentclass[11pt]{article}
\usepackage{amsmath,amssymb,amsthm}
\usepackage{fancyhdr}

% margins
\usepackage[vmargin=1in,hmargin=1.5in]{geometry}

\title{
	$A_{3}$ \\
	\large{RH 3326, Professor Freeland}
}
\author{Connor Taffe}

\begin{document}
\maketitle

Here follows my critiques of said instructions:

\section{Warnings and Cautions}

Standard manuals use two classes of safety informations, warnings and cautions. A caution is anything that could damange equipment, while a warning is anything that could injure or kill the user. Thusly, warnings must garner more attention from the reader than cautions, and both must be placed before the instruction they refer to.

\begin{enumerate}
	\item{
		The caution present at the top of the document serves only to redirect the reader. Instead, the caution should be accompanied by an attention grabbing icon and perhaps a bounding box with slightly accented background color. The information in the warning should be explicit and self contained, allowing the user to evaluate the dangers of the task at hand.
	}
	\item{
		The caution in box 1 is vague. The phrase ``when required'' is never clarified and leaves the reader wondering. This renders the caution useless.
		The document should probably also contain additional cautions aimed to protect the user's bathroom walls, tub, etc. from improper installation and therefore damage.
	}
	\item{
		Box 2 lacks a warning and seems extremely dangerous. The instructions seem to depict a man cutting a piece of aluminum rail over his knee with a hacksaw whilst straddling a bathtub. This is extremely unsafe conduct which we should not condone. Replacement of the graphic is in order, with an explicit warning and instrucitons on how to properly cut the aluminum {\it safely}.
	}
\end{enumerate}

\section{Extraneous information}

As these instructions need be consise, extraneous graphics and information should be pruned as to not distract the user.

\begin{enumerate}
	\item{
		The illustration on the right side of box 7 is useless. Use the space to include a more descriptive graphic of instruction 7. Perhaps of a pair of hands performing the action, or an exploded diagram of the materials showing the assembly that should take place.
	}
	\item{
		In the ``Tridor model only'' section, ``so that shower head does not throw water between the panels,'' the article ``the'' should be used consistently. The phrase is also out of context, and either should be excluded and only include the instructions for panel reversal, or explained. Why would the water go though the panels in one orientation and not the other? For the Tudor model, do you make sure the user installs them in the correct orientation (if they can be installed in an incorrect orientation)?
	}
\end{enumerate}

\section{Arrangement}

\begin{enumerate}
	\item{
		The contents of the kit should be near the top of the document so the user can do a quick inventory and be assured they have all the needed materials prior to embarking on the installation.
	}
	\item{
		I would recommend reorganizing these instructions into a front-and-back or two page instruction manual which would include more whitespace, headings where appropriate,  with graphics on the right and instructions on the left.
		This pattern makes it easier for the user to follow and go back and forth quickly. Furthermore, with two pages, you could expand upon your terse instruction language to add details and notes.
		Don't forget to properly label figures and reference them in the instructions.
	}
\end{enumerate}

\section{General Notes}

\begin{enumerate}
	\item{
		I would be pleased to see a wider variety of fonts used. There is very little whitespace, which is uncomfortable for the novice audiences who will be using these.
	}
	\item{
		The graphics present on this page are of low quality. Nicer, more detailed graphics should be used. They are both more asthetically pleasing, which will make our users happier, and more instructional.
	}
	\item{
		Color, although not strictly required, helps to guide users. Especially to important information like warnings and cautions. Perhaps accompanying them with a orange caution sign icon or orange or yellow borders would provide a more visible beakon.
	}
\end{enumerate}

\end{document}
