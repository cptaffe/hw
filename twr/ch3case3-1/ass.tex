\documentclass[11pt]{article}
\usepackage{amsmath,amssymb,amsthm}
\usepackage{fancyhdr}

% margins
\usepackage[vmargin=1in,hmargin=1.5in]{geometry}

\title{
	$A_{2}$ \\
	\large{RH 3326, Professor Freeland}
}
\author{Connor Taffe}

\begin{document}
\maketitle

\section{Memo}

Maloney and Brownstein,

As you know, we submitted a draft of our project report on the Omni-Tech 1000
wireless heart rate monitors to Kevin earlier this week. In a meeting with him
today, he expressed some concerns that I feel should be addressed. I think as
engineers sometimes we get too caught up in our own understanding of the project
that we forget that others who have been less directly involved may not follow.
We need to rewrite the report introduction to be more accessible. The points Kevin
brought up were some stylistic concerns about choppy writing and sometimes
awkward flow. He also mentioned breaking the introduction up into paragraphical
sections with clear topic sentences so one can navigate the page quickly.

I feel we have three distinct sections we can break this into: the problem, methods,
and result of the research. After we break it up into these three paragraphs,
we can focus on combining some sentences to create some variation in sentence
structure. We should also use a more consistent tense to produce better flow.

--Horsney

\section{Revised Introduction}

The Omni-Tech 1000 wireless heart-rate monitor consists of a belt worn around
the chest of a cyclist, which transmits a wireless heart rate signal to a
computer/reciever on the handlebars. In early April, John Horsney was approached
by the Sales and Marketing leads of Omni-Tech in regards to a technical problem
with this device. The wireless signal between the chest belt and the receiver
goes out during use, with the resulting effect that innaccurate data for the user
is presented. The sales teach is of the impression that this is a persistent problem
with this specific model. Of a total of 1620 monitors sold at \$140 each, Omni-Tech
has refunded or replaced 980 units, at a loss of \$13,720. Aside from fiscal issues,
Omni-Tech's reputation for quality and credibility has been tarnished.

Three Omni-Tech engineers, John Horsney, Tim Maloney, and Amanda Brownstein conducted
research to determine possible solutions. An initial four day period was spent
examining the current device, but concluded that simple correction was infeasable.
The problem seemed to lie in the Omni-Tech 1000's wireless technology, DSRC or Wi-Fi
802.11b. Subsequent research was focused on the consideration of alternative wireless
technologies. Criteria for this replacement was constructed, with two key points
being a requirement that the new wireless system fit into existing housing and
that it maintain current cost. By researching online resources, interviews,
field research, and product testing, research concluded successfully on April 24, 2012.

Bluetooth 4.0 Wi-Fi is Omni-Tech's optimal option for Omni-Tech 1000 product correction.
Field tests with Bluetooth observed no wireless failure. The wireless technology
comes at a \$2 discount from our existing problematic Wi-Fi solution, and provides
extra features such as multiple connections. This could permit a team leader to
receive someone’s data on an additional device. This upgrade to the current design
will correct the product's failures, provide additional features, and restore Omni-Tech's
reputation for quality and high performance equipment.

The following sections of this report will include the details of our methodology,
the results, our conclusions, and our recommendations for the ETW-2000.

\end{document}
