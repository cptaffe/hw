
\documentclass[letterpaper, 10pt]{article}

% get rid of pesky indentation
\setlength{\parindent}{0pt}

\title{Assignment II}
\author{Connor Taffe}

\begin{document}
	\maketitle

	The following is my report for Assignment II.

	\section*{Q. 1}

	First I copied the file {\tt foo-bar.c} into my directory for Assigment II.

	\begin{verbatim}
		$ cd courses/3380/ass2-2015/
		$ ls
		$ cp /tmp/foo-bar.c  .
		$ ls
		foo-bar.c
	\end{verbatim}

	Then, I compiled it using gcc with the {\tt -S} option to produce only assembly code output.

	\begin{verbatim}
		$ gcc -S foo-bar.c
		$ ls
		foo-bar.c  foo-bar.s
	\end{verbatim}

	\section*{Q. 2}

	I then used the {\tt cat} command to list the contents of {\tt foo-bar.s}, which is the file containing the generated assembly code. For brevity I have omitted many lines of output.

	\begin{verbatim}
		$ cat foo-bar.s
		.file	"foo-bar.c"
		.text
		.globl main
		.type	main, @function
		main:
		leal	4(%esp), %ecx
		andl	$-16, %esp
		pushl	-4(%ecx)
		pushl	%ebp
		... (many lines omitted)
	\end{verbatim}

	\section*{Q. 3}

	Following are the answers for the $a$, $b$, and $c$ subquestions.

	\subsection*{Sub Q. A}

		Variable $a$ is at an offset of -12 from {\tt \%ebp}, and variable $b$ is at an offset of -8 from {\tt \%ebp}. I found them via the {\tt mov} instruction that set their values from the constants 3 and 4. Since $a$ was assigned 3, and $b$, 4, I was able to find their positions.

	\subsection*{Sub Q. B}

		Variable $c$ is stored at a -4 offset. These function calls are using the System V ABI, as the caller is responsible for stack cleanup. When {\tt foo} calls {\tt bar}, {\tt bar}'s return value is stored in {\tt \%eax}, this is moved to a -4 offset from {\tt \%ebp}.

	\subsection*{Sub Q. C}

	Variable $d$ is at an offset of -4 from {\tt \%ebp}. The variable $d$ is set frop the squaring of $x$, so I looked for a multiplication of the same reference from the stack by itself, this was saved to a -4 offset of {\tt \%ebp}, so that must be $d$.

	\section*{Q. 4}

	I compiled the code as follows:

	\begin{verbatim}
		$ gcc -g -o foo-bar foo-bar.c
		$ ls
		foo-bar  foo-bar.c  foo-bar.s
	\end{verbatim}

	\section*{Q. 5}

	I then used gdb in emacs to trace the execution of the program at the assembly code level as follows:

	\begin{verbatim}
		$ emacs -nw
		(emacs fills the terminal with its new window)
	\end{verbatim}

	After emacs loads, I used the meta-x key combo to bring up a M-x prompt.

	\begin{verbatim}
		M-x gdb
		Run gdb (like this): gdb foo-bar
		(terminal refreshes)
		(gdb)
	\end{verbatim}

	Following are the answers to subquestions $a$ through $q$.

	\subsection*{Sub Q. A}

	I then listed the main function and set a breakpoint at {\tt w -= foo(a, b);}.

	\begin{verbatim}
		(gdb) list
		1       int w;
		2       int foo(int, int);
		3       int bar(int);
		4
		5       int main()
		6       {
		7         int a = 3;
		8         int b = 4;
		9         w = foo(a, b);
		10      }
		(gdb) break 9
		Breakpoint 1 at 0x8048393: file foo-bar.c, line 9.
	\end{verbatim}

	\subsection*{Sub Q. B}

	I then ran the program until it stopped at a the set breakpoint.

	\begin{verbatim}
		(gdb) r
		Starting program: /home/cptaffe/courses/3380/ass2-2015/foo-bar

		Breakpoint 1, main () at foo-bar.c:9
	\end{verbatim}

	Following is the accompanying output from the second screen.

	\begin{verbatim}
	int w;
	int foo(int, int);
	int bar(int);

	int main()
	{
	  int a = 3;
	  int b = 4;
	=>w = foo(a, b);
	}
	\end{verbatim}

	\subsection*{Sub Q. C}

	I then used {\tt disas} to disassemble the code in {\tt main()}.

	\begin{verbatim}
		(gdb) disas
		... (scrolled away)
		0x0804837e <main+10>:   push   %ebp
		0x0804837f <main+11>:   mov    %esp,%ebp
		0x08048381 <main+13>:   push   %ecx
		0x08048382 <main+14>:   sub    $0x24,%esp
		0x08048385 <main+17>:   movl   $0x3,-0xc(%ebp)
		0x0804838c <main+24>:   movl   $0x4,-0x8(%ebp)
		0x08048393 <main+31>:   mov    -0x8(%ebp),%eax
		0x08048396 <main+34>:   mov    %eax,0x4(%esp)
		0x0804839a <main+38>:   mov    -0xc(%ebp),%eax
		0x0804839d <main+41>:   mov    %eax,(%esp)
		0x080483a0 <main+44>:   call   0x80483b3 <foo>
		0x080483a5 <main+49>:   mov    %eax,0x8049634
		0x080483aa <main+54>:   add    $0x24,%esp
		0x080483ad <main+57>:   pop    %ecx
		0x080483ae <main+58>:   pop    %ebp
		0x080483af <main+59>:   lea    -0x4(%ecx),%esp
		0x080483b2 <main+62>:   ret
		End of assembler dump.
	\end{verbatim}

	\subsection*{Sub Q. D}

	Then I printed the contents of {\tt \%eip}.

	\begin{verbatim}
		(gdb) print $eip
		$2 = (void (*)()) 0x8048393 <main+31>
	\end{verbatim}

	\subsection*{Sub Q. E}

	I then ran the next few instructions up to {\tt call foo}.

	\begin{verbatim}
		(gdb) stepi
		(gdb) stepi
		(gdb) print $eip
		$1 = (void (*)()) 0x8048396 <main+34>
		(gdb) stepi
		(gdb) stepi
		(gdb) print $eip
		$2 = (void (*)()) 0x804839d <main+41>
		(gdb) stepi
		(gdb) print $eip
		$3 = (void (*)()) 0x80483a0 <main+44>
	\end{verbatim}

	\subsection*{Sub Q. F}

	Following is the current stack diagram:

	% hacky hack hack, not dealing with floats.
	\vspace{1em}
	\begin{tabular} { | c | }
		variable $b$ as passed to {\tt foo} $\longleftarrow$ {\tt esp} \\ \hline
		variable $a$ as passed to {\tt foo} \\ \hline
		... (20 bytes) \\ \hline
		local variable $b$ \\ \hline
		local variable $a$ \\ \hline
		register {\tt ecx} \\ \hline
		register {\tt ebp} $\longleftarrow$ {\tt ebp} \\
		\hline
	\end{tabular}

	\subsection*{Sub Q. G}

	I then stepped into the {\tt foo} function and used {\tt disas} to disassemble the code.

	\begin{verbatim}
		(gdb) step
		(could be more ouput here, but it was scrolled away)
		foo (x=3, y=4) at foo-bar.c:15
		(gdb) disas
		Dump of assembler code for function foo:
		0x080483b3 <foo+0>:     push   %ebp
		0x080483b4 <foo+1>:     mov    %esp,%ebp
		0x080483b6 <foo+3>:     sub    $0x18,%esp
		0x080483b9 <foo+6>:     mov    0x8(%ebp),%eax
		0x080483bc <foo+9>:     mov    %eax,(%esp)
		0x080483bf <foo+12>:    call   0x80483d1 <bar>
		0x080483c4 <foo+17>:    mov    %eax,-0x4(%ebp)
		0x080483c7 <foo+20>:    mov    0xc(%ebp),%edx
		0x080483ca <foo+23>:    mov    -0x4(%ebp),%eax
		0x080483cd <foo+26>:    sub    %edx,%eax
		0x080483cf <foo+28>:    leave
		0x080483d0 <foo+29>:    ret
		End of assembler dump.
	\end{verbatim}

	\subsection*{Sub Q. H}

	Then I ran instructions up to the {\tt call bar} instruction.

	\begin{verbatim}
		(gdb) print $eip
		$7 = (void (*)()) 0x80483b9 <foo+6>
		(gdb) stepi
		(gdb) stepi
		(gdb) print $eip
		$8 = (void (*)()) 0x80483bf <foo+12>
	\end{verbatim}

	\subsection*{Sub Q. I}

	The following is the current stack in diagram form.

	% hacky hack hack, not dealing with floats.
	\vspace{1em}
	\begin{tabular} { | c | }
		$\longleftarrow$ {\tt esp} \\ \hline
		\\ \hline
		\\ \hline
		\\ \hline
		\\ \hline
		variable $x$, passed to {\tt bar} \\ \hline
		register {\tt \%ebp} $\longleftarrow$ {\tt ebp} \\ \hline
		passed variable $x$ \\ \hline
		passed variable $y$ \\
		\hline \hline
		(main) \\
	\end{tabular}

	\subsection*{Sub Q. J}

	I then stepped into the {\tt bar} function, dand used the {\tt disas} command to show the disassembly.

	\begin{verbatim}
		(gdb) step
		bar (z=3) at foo-bar.c:22
		(gdb) disas
		Dump of assembler code for function bar:
		0x080483d1 <bar+0>:     push   %ebp
		0x080483d2 <bar+1>:     mov    %esp,%ebp
		0x080483d4 <bar+3>:     sub    $0x10,%esp
		0x080483d7 <bar+6>:     mov    0x8(%ebp),%eax
		0x080483da <bar+9>:     imul   0x8(%ebp),%eax
		0x080483de <bar+13>:    mov    %eax,-0x4(%ebp)
		0x080483e1 <bar+16>:    mov    -0x4(%ebp),%eax
		0x080483e4 <bar+19>:    leave
		0x080483e5 <bar+20>:    ret
		End of assembler dump.
	\end{verbatim}

	\subsection*{Sub Q. K}

	I then ran the instructions in {\tt bar} up to the {\tt leave} instruction.

	\begin{verbatim}
		(gdb) print $eip
		$9 = (void (*)()) 0x80483d7 <bar+6>
		(gdb) stepi
		(gdb) stepi
		(gdb) stepi
		(gdb) stepi
		(gdb) print $eip
		$10 = (void (*)()) 0x80483e4 <bar+19>
	\end{verbatim}

	\subsection*{Sub Q. L}

	The following is the stack diagram at this point.

	% hacky hack hack, not dealing with floats.
	\vspace{1em}
	\begin{tabular} { | c | }
		$\longleftarrow$ {\tt esp} \\ \hline
		\\ \hline
		\\ \hline
		local variable $d$ \\ \hline
		register {\tt \%ebp} $\longleftarrow$ {\tt epb} \\ \hline
		passed variable $z$ \\
		\hline \hline
		(foo) \\
	\end{tabular}

	\subsection*{Sub Q. M, N, O}

	I then executed the {\tt leave} instruction, the following is the stack at this point.

	% hacky hack hack, not dealing with floats.
	\vspace{1em}
	\begin{tabular} { | c | }
		passed variable $z$ $\longleftarrow$ {\tt esb}, {\tt epb} \\
		\hline \hline
		(foo) \\
	\end{tabular}
	\vspace{1em}

	I then executed the {\tt ret} instruction.

	\subsection*{Sub Q. P}

	I then ran the {\tt disas} command and viewed the disassembly.

	\begin{verbatim}
		(gdb) disas
		Dump of assembler code for function foo:
		0x080483b3 <foo+0>:     push   %ebp
		0x080483b4 <foo+1>:     mov    %esp,%ebp
		0x080483b6 <foo+3>:     sub    $0x18,%esp
		0x080483b9 <foo+6>:     mov    0x8(%ebp),%eax
		0x080483bc <foo+9>:     mov    %eax,(%esp)
		0x080483bf <foo+12>:    call   0x80483d1 <bar>
		0x080483c4 <foo+17>:    mov    %eax,-0x4(%ebp)
		0x080483c7 <foo+20>:    mov    0xc(%ebp),%edx
		0x080483ca <foo+23>:    mov    -0x4(%ebp),%eax
		0x080483cd <foo+26>:    sub    %edx,%eax
		0x080483cf <foo+28>:    leave
		0x080483d0 <foo+29>:    ret
		End of assembler dump.
	\end{verbatim}

	\subsection*{Sub Q. Q}

	\begin{itemize}
		\item[i.] The current function is {\tt foo}.
		\item[ii.] \verb!0x080483c4 <foo+17>:    mov    %eax,-0x4(%ebp)!
		\item[iii.] The {\tt ret} instruction sets the instruction pointer to the value dictated by the stack, or the value pushed there by {\tt call}.
	\end{itemize}

\end{document}
