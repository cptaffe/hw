
% Copyright (c) 2015 Connor Taffe <cpaynetaffe@gmail.com>

\input macros.tex % include macro definitions

\set(subtitlefntsize, 10pt)

\title{Assignment I}{Connor Taffe} % title

% nonnumbered section
\sec{Sec. 2}{

Here follows the report done for Assignment I about Laboratory I. The part A-D sections as follows contain abreviated promts expressed as ``\code{\$}'' to avoid uneccessary length and complexity.

% subsections
\subsec{Part A.}{
% had to manually break the filename here.
I created a new course directory named \code{courses\slash 3380\slash ass1-2015} and copied \code{\slash tmp\slash nachos-3.4-\break UALR.tgz} into it. The following screen dump shows an empty directory was created.

% code paragraphs
\codeblock{
\$ ls \par
\$ mkdir -p courses/3380/ass1-2015 \par
\$ cd courses/3380/ass1-2015/ \par
\$ ls \par
\$ cp /tmp/nachos-3.4-UALR.tgz .
}

} \subsec{Part B.}{
I then used the \code{tar} command to untar the nachos file in the working directory, and removed the \code{.tgz} file.

% code paragraphs
\codeblock{
\$ tar -xzf nachos-3.4-UALR.tgz \par
\$ ls \par
nachos-3.4  nachos-3.4-UALR.tgz \par
\$ rm nachos-3.4-UALR.tgz \par
\$ ls \par
nachos-3.4
}

} \subsec{Part C.}{
I then compiled the \code{NachOS threads/} directory as follows using the make command:

\codeblock{
\$ cd nachos-3.2/code/threads/ \par
\$ make \par
... (many lines omitted)
}

} \subsec{Part D.}{
Following compilation, I ran the resulting binary using the symlink present in the \code{threads/} directory (my current working directory).

\codeblock{
\$ ./nachos \par
... (many lines omitted) \par
Cleaning up...
}

}

}\sec{Sec. 4}{

The following is the screens for tasks A-I within Section 4.

\subsec{Part A.}{

After finishing the above Sec. 2, I \code{cd}'d up to the \code{c++example} directory and \code{rm}'d the \code{stack} file.

\codeblock{
\$ cd cd ../../c++example \par
\$ ls \par
c++.log      c++.tex	      inheritstack.h  Makefile	stack.h \par
copyright.h  hello.cpp	      list.cc	      stack	templatestack.cc \par
c++.ps	     inheritstack.cc  list.h	      stack.cc	templatestack.h \par
\$ rm stack \par
\$ ls \par
c++.log      c++.tex	      inheritstack.h  Makefile	templatestack.cc \par
copyright.h  hello.cpp	      list.cc	      stack.cc	templatestack.h \par
c++.ps	     inheritstack.cc  list.h	      stack.h
}}

\subsec{Part B.}{

Following removal of the old binary, I compiled a new one using \code{make stack}.

\codeblock{
\$ make stack \par
g++ -g -o stack stack.cc \par
... (several lines omitted) \par
\$ ls \par
c++.log      c++.tex	      inheritstack.h  Makefile	stack.h \par
copyright.h  hello.cpp	      list.cc	      stack	templatestack.cc \par
c++.ps	     inheritstack.cc  list.h	      stack.cc	templatestack.h
}}

\subsec{Part C.}{

Then I opened emacs.

\codeblock{
\$ emacs -nw \par
(new emacs window opens, controlling terminal) \par
}}

\subsec{Part D.}{

Then I used the ``meta'' (esc) and `x' keys to obtain an \code{M-x} prompt.

\codeblock{
M-x gdb \par
Run gdb (like this): gdb stack
}}

\subsec{Part E.}{

Then I listed the first few lines of stack.

\codeblock{
... (several lines omitted) \par
(gdb) list \par
... (many lines omitted) \par
128 main() \char123 \par
129 Stack *stack = new Stack(10); // Constructor with an argument.
}}

\subsec{Part F.}{

Then I set a breakpoint at line 131.

\codeblock{
(gdb) break 131 \par
Breakpoint 1 at 0x8048a26: file stack.cc, line 131.
}}

\subsec{Part G.}{

Then I ran the program, and stepped into the function \code{stack->SelfTest()}.

\codeblock{
(gdb) r \par
Starting program: /home/cptaffe/courses/3380/ass1-2015/nachos-3.4/c++example/st\
ack \par
\par
Breakpoint 1, main () at stack.cc:131 \par
(gdb) step \par
Stack::SelfTest (this=0x804a008) at stack.cc:108
}}

\subsec{Part H.}{

Then I incremented through the statements in the program, printing values of \code{count}. First the uninitialized value, and then three values in the loop.

\codeblock{
(gdb) print count \par
\$1 = 3415200 \par
(gdb) next \par
(gdb) print count \par
\$2 = 17 \par
(gdb) next \par
(gdb) \par
pushing 17 \par
(gdb) \par
(gdb) print count \par
\$3 = 18 \par
(gdb) next \par
(gdb) \par
pushing 18 \par
(gdb) print count \par
\$4 = 18 \par
(gdb) next \par
(gdb) print count \par
\$5 = 19 \par
}}

The following is the loop, the \code{=>} basically just proceeds to follow the loop logic, showing each current statement. For brevity, I have only included this one copy of the code window.

\codeblock {
// Put a bunch of stuff in the stack... \par
=>  while (!Full()) \char123 \par
cout << "pushing " << count << "\char92 n"; \par
Push(count++); \par
\char125
}

\subsec{Part I.}{

Then I used \code{gdb}'s \code{continue} command to finish tracing the program.

\codeblock{
(gdb) continue \par
... (many lines omitted) \par
popping 18 \par
popping 17 \par
\par
Program exited normally. \par
(gdb) \par
}}

}\bye % end of document
