\documentclass{article}

\usepackage{enumerate}

% Header macro
\def\AssignmentHeader#1#2#3{
  \title{
    % Assignment number
    % Provided in A sub n notation an standard english
    % using roman numerals.
    $A_{#1}$ \\ {\large Assignment \uppercase\expandafter{\romannumeral #1}} \\
    % Professor name and class.
    {\normalsize Prof. #2; #3}
  }
  \author{Connor Taffe}
}
% Configuration information
\AssignmentHeader{4}{P. Tang}{Computer Organization I}

\begin{document}
  \maketitle

  \begin{enumerate}
    % Describe the design, implementation and testing of the
    % task (b) of lab 11 as follows:

    % 1. Show the block diagram and function table of your ALU.
    \item{}

    % 2. Show the internal circuits of your ALU.
    \item{}

    % 3. Show the block diagram of your datapath and
    % list all the control points.
    \item{}

    % 4. Show the circuit of you Datapath.
    \item{}

    % 5. Show the control points activation table and derive
    % the Boolean equations of all control points.
    \item{}

    % 6. Show the internal circuit of your control unit.
    \item{}

    % 7. Testing your register transfer system under the
    % control of clock as follows:
    \item{
      \begin{enumerate}[(a)]

        % (a) Set the registers X, Y and Z to be hex 4, 1 and 7, respectively
        \item{}

        % (b) Show your testing circuit.
        \item{}

        % (c) Calculate the register value changes if we perform α, β, γ,
        % δ, ǫ and θ operations at 6 consecutive rising edges of G.
        \item{}

        %(d) Show the waveform showing that the system performs α, β, γ,
        % δ, ǫ and θ, at 6 consecutive rising edges of G to confirm that
        % the system does make the register value changes as you calculated
        % in 7(c). You need to annotate the waveform to show these 6 changes.
        \item{}

      \end{enumerate}
    }
  \end{enumerate}

\end{document}
