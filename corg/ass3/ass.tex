\documentclass[11pt]{article}
\usepackage{amsmath,amssymb,amsthm}
\usepackage{fancyhdr}
\usepackage{enumerate}
\usepackage{graphicx}

% margins
\usepackage[vmargin=1in,hmargin=1.5in]{geometry}

% Config
%%%%%%%%%%%%%%%%%%%%%%%%%%%%%%%%%%%
\newcommand{\ass}{2}
\newcommand{\name}{Connor Taffe}
\newcommand{\tno}{3742} % last 4 digits of T number.
%%%%%%%%%%%%%%%%%%%%%%%%%%%%%%%%%%%

\title{
	$A_{\ass}$ \\
	{\large Assignment \ass\\
	CS 3482; Professor Tang}
}
\author{
	\name. T no. \tno
}

\pagestyle{fancy}
\rhead{Homework \ass}
\lhead{{\name}. T no. \tno}

\begin{document}
\maketitle

\begin{enumerate}
\item{
	{\bf Question:}

	Construct and package D Flip-flop (call it {\tt D-FF}) and D Latch
	(call it {\tt Dlatch}).
	Compare the functions of your {\tt D-FF} and D-latch using the following
	circuit\footnote{
		Question related diagrams and graphics have been omitted. \label{qfoot}
	}.
	Show your test circuit and a wave form that demonstrates that
	\begin{enumerate}[(a)]
		\item{
			{\tt D-FF} changes its value {\tt Q(FF)} from 0 to 1 and 1 to 0 at
			rising edges of $D$,
		}\item{
			D-latch changes its values {\tt Q(Latch)} from 0 to 1 and 1 to 0
			(following $D$) when $G = 1$.
		}
	\end{enumerate}
	You need to annotate on the wave form for the changes above.

	{\bf Answer:}



	Insert Answer here
}

\item{
	{\bf Question:}

	Build and package a 4-bit register (call it {\tt Reg-4}).
	Test your 4-bit register using the following
	circuit\footnote{See footnote \ref{qfoot}}.
	Show your simulation to demonstrate that your register can transfer its
	data to another one at the same time it receives new data.

	You need to build and package a quadruple 2-to-1 multiplexer without
	enable control first.
	Name it as {\tt 4mux-2} with the corresponding pin names in the circuit.

	Show your test circuit and a wave form to demonstrate that
	registers A, B and C store and rotate
	hex B, 6 and A at each rising edges of {\tt CLK}.
	You need to annotate on the wave form for the changes of registers.

	{\bf Answer:}

	Insert Answer here
}

\item{
	{\bf Question:}

	Design and implement a 4-bit up counter (call it {\tt UCT4}) as a
	finite-state machine using D flip-flops following the steps as follows:
	\begin{enumerate}[(a)]
		\item {
			Draw the truth tables for $D_3$, $D_2$, $D_1$, $D_0$ of the flip-flops in terms of
			$Q_3$, $Q_2$, $Q_1$, $Q_0$ of the previous cycle.
		} \item {
			Use Karnaugh map to find the simplest sum-of-product equations for
			$D_3$, $D_2$, $D_1$, $D_0$.
		} \item {
			Implement the counter using the equations obtained and show the internal
			circuit of the counter.
		} \item {
			Build a circuit to test your {\tt UCT4}. Show the test circuit and a wave form to
			demonstrate that
		}
	\end{enumerate}
	\begin{enumerate}[i.]
		\item {
			The counter {\tt UCT-4} can increase the counter output from hex 0 to F and
			goes back to 0 after F.
		} \item {
			The $Q_3$, $Q_2$, $Q_1$ and $Q_0$ changes values at the same time
			when the output changes from F to 0.
		}
	\end{enumerate}

	You need to annotate on the wave form for the two points above.

	{\bf Answer:}

	Insert Answer here
}

\item{
	{\bf Question:}

	Design and implement a 4-bit up counter with parallel load
	(call it {\tt UCT4-LD}) based on the up-counter of finite-state
	machine you built above with the following pins:
	\begin{itemize}
		\item {
			{\tt CLR} for the active-low clear to set the counter to all 0.
		} \item {
			$G$ for the input to change the counter value at the rising edge of clock cycle
		} \item {
			$D_{3−0}$ for the parallel load data input 1
		} \item {
			{\tt LD} to control the parallel data load.
			When $\text{\tt LD} = 1$, the counter receives the parallel
			data from $D_{3−0}$ at the rising edge of $G$.
			When $\text{\tt LD} = 0$, the counter value is incremented
			at the rising edge of $G$.
		} \item {
			$Q_{3−0}$ to output the 4-bit value of the counter
		}
	\end{itemize}

	You need to use the quadruple 2-to-1 multiplexer you built in lab 4.
	\begin{enumerate}[(a)]
		\item {
			Build the {\tt UCT4-LD} and show the internal circuit of the counter.
		} \item {
			Build a circuit to test your {\tt UCT4-LD}.
			Show the test circuit and a wave form to demonstrate that
			\begin{enumerate}[i.]
				\item {
					The counter {\tt UCT4-LD} can increase the counter output from hex
					0 to 5, then load hex A to the counter, and then switch to the counting
					mode to count from A to F and then to 0.
				}
			\end{enumerate}
		}
	\end{enumerate}

	You need to annotate on the wave form for the time of each change.

	{\bf Answer:}

	Insert Answer here
}

\item{
	{\bf Question:}

	Build and package a 16x8 RAM memory array called {\tt RAM16x8}
	using D-latches and package.
	Show the internal circuit of your {\tt RAM16x8}. (Use the reduced size).

	{\bf Answer:}

	Insert Answer here
}

\item{
	{\bf Question:}

	Use the following test circuit\footnote{See footnote \ref{qfoot}} to test your 16x8 bi-directional memory system.

	Show your test circuit and a wave form of test that writes hex 37 and A7 to
	memory locations 7 and 8, respectively, and then reads locations 7 and 8.

	You need to annotate on the wave form for the relevant times of these writings
	and readings.

	{\bf Answer:}

	Insert Answer here
}

\item{
	{\bf Question:}

	Then remove all the input devices and replace them with input and bi-directional
	pins and package and call it {\tt MyMemoryChip}.
	\begin{enumerate}[(a)]
		\item {
			Show the internal circuit of {\tt MyMemoryChip}.
		} \item {
			Test your {\tt MyMemoryChip} with the following
			circuit\footnote{See footnote \ref{qfoot}}.
		}
	\end{enumerate}

	Show your test circuit and a wave form of the test that writes hex 8C and
	6E to location 9 and A, respectively, and then reads locations 9 and A.

	You need to annotate on the wave for the relevant times of the writings and
	readings.

	{\bf Answer:}

	Insert Answer here
}

\end{enumerate}
\end{document}
