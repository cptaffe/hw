\documentclass[11pt]{article}
\usepackage{amsmath,amssymb,amsthm}
\usepackage{fancyhdr}
\usepackage{enumerate}
\usepackage{graphicx}

% margins
\usepackage[vmargin=1in,hmargin=1.5in]{geometry}

% Config
%%%%%%%%%%%%%%%%%%%%%%%%%%%%%%%%%%%
\newcommand{\ass}{2}
\newcommand{\name}{Connor Taffe}
\newcommand{\tno}{3742} % last 4 digits of T number.
%%%%%%%%%%%%%%%%%%%%%%%%%%%%%%%%%%%

\title{
	$A_{\ass}$ \\
	{\large Assignment \ass\\
	CS 3482; Professor Tang}
}
\author{
	\name. T no. \tno
}

\pagestyle{fancy}
\rhead{Homework \ass}
\lhead{{\name}. T no. \tno}

\begin{document}
\maketitle

\begin{enumerate}
\item {
	To test the circuits, we will assure that the $Q$ only changes when $G$
	is 1 and $D$ is changed. $Q$ should reflect the value of $D$ while $G$
	is on, and then lock when $Q$ is 0. $G$ should serve as the {\it latch}
	in the D latch circuit.

	The D flip flop is edge triggered rather than level triggered as the
	D latch is. This means that the D flip flop should only change on rising
	edges of $G$. $Q$ only reflects $D$ if it is at the rising edge of $G$,
	otherwise it is locked.
}
\item{}
\item{
	Each row of output $D$ bits is the input $Q$ bits of the last state
	plus 1. The sum of minterms obtained from the Karnaugh maps of each of
	the four $D$s are as follows:

	\[
		D_{0}=Q_{0}'
	\] \[
		D_{1}=Q_{0}'Q_{1}+Q_{1}Q_{1}'
	\] \[
		D_{2}=Q_{3}'Q_{2}'Q_{1}Q_{0}+Q_{3}'Q_{2}Q_{1}'+Q_{3}'Q_{2}Q_{1}Q_{0}'
			+Q_{3}Q_{2}Q_{1}Q_{0}+Q_{3}Q_{2}'Q_{1}'+Q_{3}Q_{2}'Q_{1}Q_{0}'
	\] \[
		D_{3}=Q_{3}Q_{2}+Q_{3}Q_{2}'Q_{1}'+Q_{3}Q_{2}'Q_{1}Q_{0}'
			+Q_{3}'Q_{2}Q_{1}Q_{0}
	\]
}

\end{enumerate}
\end{document}
