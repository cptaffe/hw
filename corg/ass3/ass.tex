\documentclass[11pt]{article}
\usepackage{amsmath,amssymb,amsthm}
\usepackage{fancyhdr}
\usepackage{enumerate}
\usepackage{graphicx}

% margins
\usepackage[vmargin=1in,hmargin=1.5in]{geometry}

% Config
%%%%%%%%%%%%%%%%%%%%%%%%%%%%%%%%%%%
\newcommand{\ass}{3}
\newcommand{\name}{Connor Taffe}
\newcommand{\tno}{3742} % last 4 digits of T number.
%%%%%%%%%%%%%%%%%%%%%%%%%%%%%%%%%%%

\title{
	$A_{\ass}$ \\
	{\large Assignment \ass\\
	CS 3482; Professor Tang}
}
\author{
	\name. T no. \tno
}

\pagestyle{fancy}
\rhead{Homework \ass}
\lhead{{\name}. T no. \tno}

\begin{document}
\maketitle

\begin{enumerate}
\item {
	To test the circuits, we will assure that the $Q$ only changes when $G$
	is 1 and $D$ is changed. $Q$ should reflect the value of $D$ while $G$
	is on (1), and then lock when $G$ is off (0). $G$ should serve as the {\it latch}
	in the D latch circuit.

	The D flip flop is edge triggered rather than level triggered as the
	D latch is. This means that the D flip flop should only change on rising
	edges of $G$. $Q$ only reflects $D$ if it is at the rising edge of $G$,
	otherwise it is locked.
}
\item{}
\item{
	Each row of output $D$ bits is the input $Q$ bits of the last state
	plus 1. The sum of minterms obtained from the Karnaugh maps of each of
	the four $D$s are as follows:

	\[
		D_{0}=Q_{0}'
	\] \[
		D_{1}=Q_{1}'Q_{0}+Q_{1}Q_{0}'
	\] \[
		\begin{aligned}
		D_{2}&=Q_{3}'Q_{2}'Q_{1}Q_{0}+Q_{3}'Q_{2}Q_{1}'+Q_{3}'Q_{2}Q_{1}Q_{0}'
			+Q_{3}Q_{2}Q_{1}Q_{0}+Q_{3}Q_{2}'Q_{1}'+Q_{3}Q_{2}'Q_{1}Q_{0}' \\
			& = Q_{3}'(Q_{2}'Q_{1}Q_{0}+Q_{2}Q_{1}'+Q_{2}Q_{1}Q_{0}')
				+ Q_{3}(Q_{2}'Q_{1}Q_{0}+Q_{2}Q_{1}'+Q_{2}Q_{1}Q_{0}') \\
			& = (Q_{3}'+Q_{3})(Q_{2}'Q_{1}Q_{0}+Q_{2}Q_{1}'+Q_{2}Q_{1}Q_{0}') \\
			& = \top(Q_{2}'Q_{1}Q_{0}+Q_{2}Q_{1}'+Q_{2}Q_{1}Q_{0}') \\
			& = Q_{2}'Q_{1}Q_{0}+Q_{2}Q_{1}'+Q_{2}Q_{1}Q_{0}' \\
		\end{aligned}
	\] \[
		D_{3}=Q_{3}Q_{2}'+Q_{3}Q_{2}Q_{1}'+Q_{3}Q_{2}Q_{1}Q_{0}'
			+Q_{3}'Q_{2}Q_{1}Q_{0}
	\]

	Note I did some quick extra algebra to save myself some duplicate gates. As
	the $Q_{3}$s factor out and become a tautology because a thing or it's inverse
	must be true.

	For the resulting circuit, I've built one with $G$ and $CLR$ inputs, and $D$
	and $Q$ outputs, as per your instructions that they must be testable. The $D$
	output in this circuit is unrelated to the $D$ inputs in the next question.

	For each clock cycle you can see that on the rising edge of the input clock
	$G$, the counter increments it's four bit internal state. The finite state
	machine chugs along preserving its state in the D flip flop memory units. When
	the previous state is hexadecimal F or 16, all the bits are used up and the
	state returns to 0 as per the truth table I originally came up with the sum
	of minterms specified with.
}
\item{
	This version of the circuit adds a quadruple two to one muliplexer which
	exposes a $LD$ switch to the circuit. When on, this switches the current state
	from the calculated next state from the state machine logic, to one provided
	by a user, allowing the counter to be ``set.''

	Testing reaveals a fully
	functioning implementation which can take a user provided number and count
	from it, properly wrapping as it would with its own counter.
}

\end{enumerate}
\end{document}
