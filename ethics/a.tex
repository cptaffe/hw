\documentclass{amsart}

\title{$a_{8}$: The End of an Era}
\author{Connor Taffe}

\begin{document}
  \maketitle

  The economy of the world has been advancing towards the use of tools to complete tasks since the beginning of recorded history. Humans use tools to increase their efficiency and decrease the amount of time and effort required to complete a task. All of mans possessions and products can be thought of as tools that do or help some task previously more difficult. As man has created tools, jobs that people formerly did have been replaced by them.

  Job replacement has lead to an influx of people into areas that either did not previously exist or are not yet replaced by a tool. The most recent example of such large scale job replacement or augmentation is the industrial revolution. Introducing machines such as the cotton gin replaced or augmented previously avaliable jobs. Society at large, though, benefitted greatly from this improvement.

  In general, a trend of machines and computers, tools, replacing low skill jobs and thusly pushing society into more knowledgable and higher educated roles has been quite pervasive in the last century. But, regardless of this, the unemployment rate now is not unsimilar to what it was a century ago, and the technological advancements made since then are dramatic.

  The computer revolution is the newest incarnation of an economic revlution, moving machines from large, metal contraptions to tiny transistorized maps of decisions. These tiny machines allowed for all the mechanical logic in a large, complex, expensive machine to be cheaply replicated on a tiny circuit board. As computers advanced, and moore's law took hold, the amount of decisions and information you could represent in some space on a chip has been doubling. This exponential growth of computational power has made it a universal tool for almost any problem. Accounting software has replace accountants, automatic care diagnostics has replaced some mechanic positions. But new jobs have also been created. An entire technology industry has sprung up all over the world.

  This new technology industry employs huge swaths of people doing programming, manufacturing, and managerial tasks. It also employs specialists in many areas to assist in the computerization of tasks. Google and Microsoft, for instance, employs economists, engineers of all specialties, and mathematicians as well as computer scientists to produce products like self-driving cars, breakthrough medical research, ``smart'' thermostats, ``smart'' phones, laptops, web browsers. Just the mention of a web browser invokes all the gigantic change that the internet has brought on top of the direct impact of computation. Video and audio media can produce huge streams of revenue. Firms like Neflix not only employ computer scientists, but artists. Netflix, Hulu, and the like have financed many of our most popular shows and movies.

  Humans are made of cells that depend on deltas as a source of energy. It requires a ``voltage'' in the form of food and materials to produce other inter-cellular deltas, etc. Outside of the cell, humans still thrive on change. The economic ``voltage'' is how well our society is doing. From outside the system there are few inputs or outpus, but the pool of money stirs and produces economic satisfaction. As our economy and culture advances, more and more of the money will belong to industries we never before imagined. The U.S. and many other nations are moving more and more to a service economy, producing and consuming services as our pot-stirring action. As technology replaces much of manufacturing and business, there will always be technology-maintainers, artists, etc. A computer will never make you an artisan latte. McDonald's may automate, but my favourite coffee shop will always be stocked by humans, as my movies will always be imagined by humans, as my art will always be made by humans.

  In conclusion, as our economy advances to be less dependent on human intervention and action, our economy will just move further up Mazlow's heirarchy. Providing emotional support, food, thought provoking media, and entertainment.
\end{document}
