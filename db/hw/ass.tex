\documentclass[11pt]{article}
\usepackage{amsmath,amssymb,amsthm}
\usepackage{fancyhdr}
\usepackage{hw}

% margins
\usepackage[vmargin=1in,hmargin=1.5in]{geometry}

% Config
%%%%%%%%%%%%%%%%%%%%%%%%%%%%%%%%%%%
\newcommand{\hw}{2}
\newcommand{\name}{Connor Taffe}
\newcommand{\tno}{3742} % last 4 digits of T number.
%%%%%%%%%%%%%%%%%%%%%%%%%%%%%%%%%%%

\title{
    $H_{\hw}$ \\
    {\large Homework Assignment \rom{\hw}}
}
\author{
    \name. T no. \tno
}
\date{March $10^{\text{th}}$, 2015}

\pagestyle{fancy}
\rhead{Homework {\rom{\hw}}}
\lhead{{\name}. T no. \tno}

\begin{document}
\maketitle

\begin{question}

    Given the following relation schema R(A, B, C, D, E) and the
database, answer the following questions (diagram not reproduced).

\begin{subquestion}
    % restate the question
    List all super keys of $R$.

    % answer the question
    Attributes $C$ and $D$ are candidate keys of relation $R$ becuase they are unique to each tuple in $R$.

    \begin{align}
        \text{where } K\in\left\{C, D\right\}, \text{ and } t \text{ is a tuple }\in{R}\\
        t_{1}[K] \ne t_{2}[K] \ne t_{3}[K] \ne t_{4}[K]\\
        \therefore \forall t\in{R}, t_{n}[K] \ne t_{m}[K]
    \end{align}

    Since we now have $C$ and $D$ as candidate keys, we can narrow the list of super keys to the tuples as follows:

    Our minimal superkeys, or keys are

    \begin{equation}
        (C), (D)
    \end{equation}

    Other superkeys are any set of $C$ or $D$ with any other key (including $C$ or $D$) as long as the key.

    \begin{equation}
        \left\{ K\subset R \mid (C \lor D)\cup K \right\}
    \end{equation}

    Meaning the set of $C$ or $D$ or both, unioned with any set of other elements, including the null set, $\phi$.

    The complete list is as follows:
    \begin{align*}
        (C), (D), (C, D),
        % C combos
        (A, B, C, E),
        (A, C), (B, C), (C, E),
        (A, B, C),\\ (A, C, E),
        (B, C, E),
        % D combos
        (A, B, D, E),
        (A, D), (B, D), (E, D),
        (A, B, D),\\ (A, D, E),
        (B, D, E),
        % C D combos
        (A, B, C, D, E),
        (A, C, D), (B, C, D),\\ (C, D, E),
        (A, B, C, D), (A, C, D, E),
        (B, C, D, E)\\
    \end{align*}
\end{subquestion}

\begin{subquestion}
    % restate the question
    List all keys of $R$.

    % answer the question
    As indicated in the prelude of Q. 1.1, $C$ and $D$ are the keys of $R$ because keys are defined as superkeys that cannot have an element removed and remain a superkey.
\end{subquestion}

\begin{subquestion}
    % restate the question
    List all candidate keys of $R$.

    % answer the question
    As indicated in the prelude of Q. 1.1, $C$ and $D$ are the candidate keys of $R$. As candidate keys are defined as a key in the case where a relational schema has more than one key, both keys are candidate keys.
\end{subquestion}

\begin{subquestion}
    % restate the question
    Suggest a primary key for $R$.

    % answer the question
    Both candidate keys $C$ and $D$ will work as primary keys for $R$.
\end{subquestion}

\end{question}

\begin{question}
    Short Answers

\begin{subquestion}
    Explain the differences among an entity, an entity type, and an entity set.

    An entity is one record or instance with certain attributes. An entity type is a set of entities that share the same attributes. An entity set is all the elements of the same entity type in the database at any point.

    \begin{align}
        \label{eqEntity}
        e_0 &\in E_t, \text{ where $E_t$ is the entity type.} \\
        \label{eqEntitySet}
        E_s &= E_t \cap D, \text{ where $D$ is the database.}
    \end{align}

    Here \ref{eqEntity} shows an entity $e_0$ as an element of an entity type $E_t$ while \ref{eqEntitySet} shows the set of all elements of a type, $E_s$, the entity set, in the database $D$.

\end{subquestion}

\begin{subquestion}
    Explain the difference between an attribute and a value set.

    An attribute is a quality that describes an entity, while a value set is the set of possible values for an attribute. An attribute is a member of its value set.

    \begin{equation}
        a \in V_s
    \end{equation}
\end{subquestion}

\end{question}

\begin{question}
    Consider the following relational database schema (schema not reproduced).

    \begin{subquestion}
        Answer true if the statement is necessarily true; false otherwise.

        \begin{subsubquestion}
            A tuple with the {\tt NULL} value on {\tt C\_ID} can be inserted to the {\tt COURSE} relation.

            False, {\tt C\_ID} serves as the primary key (probably) for {\tt COURSE} and is used as a foriegn key elsewhere, and NULL is not a proper for a foreign key that points to a value.

        \end{subsubquestion}

        \begin{subsubquestion}
            {\tt C\_ID} in the {\tt SECTION} schema is a superkey.

            False, there would theoretically be many sections of the same course.
        \end{subsubquestion}
    \end{subquestion}

    \begin{subquestion}
        Short Answers

        \begin{subsubquestion}
            Name the foreign key(s) of the {\tt STUDENT} schema in the diagram.

            \begin{itemize}
                \item[{\tt D\_ID}]{
                    Foriegn key to the {\tt DEPARTMENT} schema.
                }
                \item[{\tt P\_ID}]{
                    Foriegn key to the {\tt PROFESSOR} schema.
                }
            \end{itemize}
        \end{subsubquestion}
        \begin{subsubquestion}
            Name the primary key of the {\tt ENROLL} schema.\

            A student can probably retake a course, meaning ({\tt S\_ID}, {\tt C\_ID}, {\tt SEC\_ID}) is the primary key.
        \end{subsubquestion}
    \end{subquestion}
\end{question}
\begin{question}
    Answer true if the statement is necessarily true; false otherwise.

    \begin{subquestion}
        The operations {\tt UNION}, {\tt INTERSECTION}, {\tt SET DIFFERENCE} and {\tt CROSS PRODUCT} require the relations on which they are applied be union compatible.

        True.
    \end{subquestion}

    \begin{subquestion}
        The operations {\tt UNION}, {\tt INTERSECTION}, and {\tt SET DIFFERENCE} are commutative.

        True.
    \end{subquestion}

    \begin{subquestion}
        Which of the following queries is true? (options not reproduced)

        $b$ is true.
    \end{subquestion}
\end{question}
\begin{question}
    Short Answers: Consider the relations $r$ and $s$ below, $r(a, b, c, d)$ and $s(e, f, g)$. (relations not reproduced)

    \begin{subquestion}
        Show the results of the operation {\tt select * from $r$ UNION select * from $s$}? If there is an error, explain why?

        This errors, the relations are not union compatible as they have differing numbers of elements.
    \end{subquestion}

    \begin{subquestion}
        Show the results of the operation, {\tt select count(*), avg($g$) from $s$}?

        It will yeild $(4, 10)$.
    \end{subquestion}

\end{question}

\begin{question}
    Consider the following tales, answer the questions in {\tt SQL}.

    \begin{subquestion}
        List the students ({\tt ID}, {\tt NAME}) for either computer science or electrical engineering majors.

        {\tt select ID, NAME from EEE\_MAJORS UNION select ID, NAME from CSE\_MAJORS;}
    \end{subquestion}

    \begin{subquestion}
        List the students ({\tt ID}, {\tt NAME}) for either computer science or electrical engineering majors.

        {\tt select NAME from TEACHES where (COURSE\_ID=`CPSC2376' OR\\ COURSE\_ID=`CPSC3375');}
    \end{subquestion}

    \begin{subquestion}
        List the students ({\tt NAME}) for computer science majors, who are not double majors in electrical engineering.

        {\tt select NAME from IEE\_MAJORS LEFT JOIN select NAME from CSE\_MAJORS;}
    \end{subquestion}

\end{question}

\end{document}
