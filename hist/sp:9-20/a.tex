
\documentclass[man,11pt,a4paper, biblatex]{apa6}

\usepackage[english]{babel}
\usepackage[hidelinks]{hyperref}
\DeclareLanguageMapping{english}{american-apa}

\title{Confucius, The Analects}
\shorttitle{Confucious}
\affiliation{University of Arkansas at Little Rock}
\author{Connor Taffe}

\begin{document}
  \maketitle

  The Analects are a collection of sayings, authored by Cofucius' early disciples and claimed as a sacred book by Confucius' later followers. Its purpose is to preserve what is considered the wisdom of Confucius, and teach his followers such.

  \section{Discussion Question}
  What is filial piety, and how does filial piety serve as the bedrock of Confucius’s philosophical system?

  Confucius' philosophy centers around contentness with ones position in society, and conformance to ``the Way.'' This more perfect philosophy, according to Confucius, is based on filial piety. One excerpt from the text shows the importance of filial piety.

  \begin{quote}
    Master You said: “Those who are filial to their parents and obedient to their elder brothers but are apt to defy their superiors are rare indeed; those who are not apt to defy their superiors, but are apt to stir up a rebellion simply do not exist. The gentleman applies himself to the roots. Only when the roots are well planted will the Way grow. Filial piety and brotherly obedience are perhaps the roots of humanity [ren], are they not?”
  \end{quote}

  Confucius compares filial piety with the roots of a plant, the metaphorical plant being ``the Way.'' Therefore, Confucius establishes filial piety as perhaps the most important component of his philosophy to a follower. As a follower, one could not ever be considered a Gentleman if he was not pious to his parents and obedient to his brother as he could not embody ``the Way.''

  As shown by Confucius' test of piousness, a three year period where one must not variate from his father's ways, his definition of piousness includes taking on all aspects of one's father's duties. This would include his position in society and his skills, which would preserve the balance of professions. This vision rests on Confucius' idea that everyone should be content with their place regardless of what it is, and that ones duty is more important than enjoyment of their task; or that ones enjoyment of a task can be controlled through following ``the Way.''

  Interestingly, Confucius encourages his followers to shelter their parents and parents to shelter their children even in the face of the law. He seems concerned more with preserving familial trust than societal trust. This most likely centers on his idea of a society where everyone inherits from their parents their position in society, and it is more important to ensure that one learns from their parents than society be completely lawful.



\end{document}
